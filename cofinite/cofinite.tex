\documentclass[]{article}
\usepackage{amssymb,amsmath}
\usepackage[utf8]{inputenc}
\usepackage{fontspec}
\setmainfont{Bitstream Charter}
\usepackage{xcolor}
\usepackage[margin = 1.5in]{geometry}
\usepackage{graphicx}
\usepackage{tikz}
\usepackage{physics}
\usepackage{amsthm}
\usepackage{mathtools}
\usepackage{authblk}
\setlength{\parindent}{0pt}
\theoremstyle{definition}
\newtheorem{theorem}{Theorem}
\newtheorem*{theorem*}{Theorem}
\newtheorem*{definition*}{Definition}
\newtheorem{prop}{Proposition}
\newtheorem{corollary}{Corollary}[theorem]
\newtheorem*{remark}{Remark}
\theoremstyle{definition}
\newtheorem{definition}{Definition}[section]
\newtheorem{lemma}{Lemma}[section]
\newtheorem{proposition}{Proposition}[section]
\newtheorem*{proposition*}{Proposition}
\newtheorem{example}{Example}[section]

\setlength{\parskip}{3pt}

\title{Preorder Cannot Generate the Cofinite Topology on Infinite Sets}
\author{Kexing Ying}
\date{April 3, 2022}

\begin{document}

\maketitle

Kevin Buzzard wanted to understand the topologies generated by preorders and asked whether or not 
the topology generated by any preorder on \(\mathbb{C}\) can equal to either the usual topology or the cofinite topology.
The first question is rather easy to one can take the base of the usual topology to be the boxes and of which
is generated by the preorder 
\[z < w \iff \text{Re}(z) < \text{Re}(w) \text{ and } \text{Im}(z) < \text{Im}(w).\]
The second part is a bit more tricky and it turns out that the following theorem is true:
\begin{theorem*}
  Given a infinite set \(X\), there does not exist a preorder on \(X\) which generates the cofinite topology.
\end{theorem*} 
We will in this short article provide a proof for this theorem.
For reference, let us first quickly recall some basic definitions from lattice theory.

\begin{definition*}[Preorder]
  For a given set \(X\), a preorder \((\le)\) on \(X\) is a binary relation such that 
  \begin{itemize}
    \item it is reflexive: for all \(x \in X\), \(x \le x\), and 
    \item it is transitive: for all \(x, y, z \in X\), if \(x \le y\) and \(y \le z\), then \(x \le z\).
  \end{itemize}
\end{definition*}

\begin{definition*}[Strict order]
  For a given set \(X\), a strict order \((<)\) on \(X\) is a binary relation such that 
  \begin{itemize}
    \item it is irreflexive: for all \(x < X\), \(\neg x < x\), and 
    \item it is transitive: for all \(x, y, z \in X\), if \(x \le y\) and \(y \le z\), then \(x \le z\).
  \end{itemize}
\end{definition*}

It is clear that every preorder has an associated strict order by setting \(x < y\) whenever \(x \le y\) and \(\neg y \le x\).
With this in mind, the topology generated the preorder \((\le)\) is defined to be is the smallest topology such that all sets of the 
form \(\{y \in X \mid y < x\}\) and \(\{y \in X \mid x < y\}\) are open for any \(x \in X\) where \((<)\) is the strict 
order associated with the preorder. 

On the other hand, the cofinite topology on \(X\) is the topology in which the open sets are precisely 
the sets whose complements are finite or the empty set.

\begin{theorem*}
  Given a infinite set \(X\), there does not exist a preorder on \(X\) which generates the cofinite topology.
\end{theorem*} 
\begin{proof}
  Suppose there exists a preorder \((\le)\) with the associated strict order \((<)\) such that the order topology generated by it is the same as the cofinite 
  topology. For the cofinite topology, any non-empty open sets must intersect, and thus, since \(\{z \mid x < z\}\) and 
  \(\{z \mid z < x\}\) are open in the order topology, their intersection is nonempty for all \(x\) if neither sets are empty. 
  But, if this is the case, there exists some \(z\) such that \(z < x < z\) implying \(z < z\) which is a contradiction. 
  Hence, at least one of the set is empty.
  
  Now, denoting \(U_+\) the set of \(x\) for which \(\{z \mid z < x\}\) is non-empty (so \(\{z \mid x < z\} = \varnothing\) for all \(x \in U_+\)), as 
  \(\{z \mid z < x\}\) is open, \(\{z \mid \neg z < x\}\) is finite for all \(x \in U_+\). As for all \(x, y \in U_+\), if \(y < x\), 
  then \(x \in \{z \mid y < z\}\) contradicting \(\{z \mid y < z\} = \varnothing\), \(\neg y < x\) for all \(x, y \in U_+\). Hence, 
  \(y \in \{z \mid \neg z < x\}\) for all \(y \in U_+\). Thus, as \(\{z \mid \neg z < x\}\) is finite, so is \(U_+\). 
  
  Similarly, defining \(U_-\) the set of \(x\) for which \(\{z \mid x < z\}\) is non-empty, by the same argument, \(U_-\) is finite. 
  Hence, \(\{z \mid x < z\}\) and \(\{z \mid z < x\}\) is empty for all but finitely many \(x\). But then, the topology generated by 
  the preorder has a finite base, implying only finitely many sets are open which is a contradiction.
\end{proof}

\end{document}
