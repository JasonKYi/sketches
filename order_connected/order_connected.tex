\documentclass[]{article}
\usepackage{amssymb,amsmath}
\usepackage[utf8]{inputenc}
\usepackage{fontspec}
\setmainfont{Bitstream Charter}
\usepackage{xcolor}
\usepackage[margin = 1.5in]{geometry}
\usepackage{graphicx}
\usepackage{tikz}
\usepackage{physics}
\usepackage{amsthm}
\usepackage{titling}
\usepackage{mathtools}
\usepackage{authblk}
\setlength{\parindent}{0pt}
\theoremstyle{definition}
\newtheorem{theorem}{Theorem}
\newtheorem*{theorem*}{Theorem}
\newtheorem{prop}{Proposition}
\newtheorem{corollary}{Corollary}[theorem]
\newtheorem*{remark}{Remark}
\theoremstyle{definition}
\newtheorem*{definition*}{Definition}
\newtheorem{lemma}{Lemma}[section]
\newtheorem{proposition}{Proposition}[section]
\newtheorem*{proposition*}{Proposition}
\newtheorem{example}{Example}[section]

\setlength{\parskip}{3pt}

\begin{document}
\title{Order Connected Sets are Measurable in \(\mathbb{R}^n\)}
\author{Kexing Ying}
\date{July 30, 2022}
\maketitle

Bollobas claimed without proof that an order connected set (with respect to the pointwise ordering) 
in \(\mathbb{R}^n\) is measurable. Yael Dillis asked to know the details of the proof and so, I wrote 
this short note.

\begin{definition*}[Order connected]
  Given a partially ordered set \((X, (\le))\), \(S \subseteq X\) is said to be order connected if for all \(x, y \in S\), 
  \[[x, y] := \{z \mid x \le z \le y\} \subseteq S.\]
\end{definition*}

Equipping \(\mathbb{R}^n\) with the point-wise ordering (i.e. for 
\(x := (x_i)_{i = 1}^n,  y := (y_i)_{i = 1}^n \in \mathbb{R}^n\), \(x \le y\) if and only if 
\(x_n \le y_n\) for all \(n = 1, \cdots, n\)), \(\mathbb{R}^n\) form a partially ordered set and we 
can therefore talk about its order connected subsets.

\begin{theorem*}
  \(S \subseteq \mathbb{R}^n\) is Lebesgue measurable if \(S\) is order connected with respect to the 
  point-wise ordering on \(\mathbb{R}^n\).
\end{theorem*} 
\begin{proof}
  Defining \(S^+ := \{x \mid \exists \ y \in S, y \le x\}\) and \(S^- := \{x \mid \exists \ y \in s, x \le y\}\), we 
  observe \(S = S^+ \cap S^-\). Therefore, it is sufficient to show that \(S^+\) and \(S^-\) are measurable. 
  We will show measurability for \(S^+\) while the case for \(S^-\) is similar. 

  It is clear that for all \(x \in \mathbb{R}^n\), \(\{x\}^+\) is measurable. So, defining \(Q := S^+ \cap \mathbb{Q}^n\),
  \[Q^+ = \bigcup_{q \in Q} \{q\}^+ \subseteq S^+\]
  is measurable. Furthermore, \(S^+ \setminus Q^+ \subseteq \partial S^+\) since for all \(x \in S^\circ\), there exists some 
  open neighborhood \(B \subseteq S^\circ\) of \(x\); so, as \(Q\) is dense in \(S^+\), there exists an element 
  \(q \in Q \cap B\) such that \(q \le x\) and hence, \(x \in \{q\}^+ \subseteq Q^+\). Now, as the Lebesgue \(\sigma\)-algebra is 
  complete and \(S^+ = Q^+ \cup (S^+ \setminus Q^+)\), it suffices to show that \(\partial S^+\) is a null-set.

  To show \(\text{Leb}(\partial S^+) = 0\) we will invoke the Lebesgue density theorem. Namely, by showing 
  \[\partial S^+ \subseteq \{x \in \overline{S^+} \mid d(x) \notin \{0, 1\}\}\]
  where 
  \[d(x) := \lim_{\epsilon \to 0} d_\epsilon(x) = \lim_{\epsilon \to 0} 
    \frac{\text{Leb}(\overline{S^+} \cap B_\epsilon(x))}{\text{Leb}(B_\epsilon(x))}\]
  as \(\overline{S^+}\) is closed and hence measurable, the Lebesgue density theorem tells us 
  \[\text{Leb}(\partial S^+) \le \text{Leb}(\{x \in \overline{S^+} \mid d(x) \notin \{0, 1\}\}) = 0.\]
  
  Indeed, for all \(x \in \partial S^+, \epsilon > 0\), \(\overline{S^+} \cap B_\epsilon(x)\) must contain 
  the with the right-upper quadrant of the ball by the very definition of \(S^+\), implying 
  \(\text{Leb}(\overline{S^+} \cap B_\epsilon(x)) \gtrapprox 2^{-n} \text{Leb}(B_\epsilon(x))\) bounding 
  \(d(x)\) from below. Similarly, \(\overline{S^+}^c \cap B_\epsilon(x)\) must contain the left-bottom 
  quadrant of the ball as otherwise \(x\) is contained by some \(\{y\}^+\) for some \(y \in \overline{S}^+\) 
  contradicting \(x \in \partial S^+\). Hence, \(\text{Leb}(\overline{S^+} \cap B_\epsilon(x)) \lessapprox 
  (1 - 2^{-n}) \text{Leb}(B_\epsilon(x))\) bounding \(d(x)\) from above and so, proving the required 
  inclusion.
\end{proof}

\end{document}
